\documentclass[a4paper,titlepage]{article}

\begin{document}

\title{Technical Specification\\
       \emph{Zarmina}}
\author{Kieren Davies}
\date {22 August 2012}
\maketitle

\section{Executive summary}

\section{Style and theme}
economically oriented
real time
base building
survival

\section{Story}

The premise of the game will be based around early human colonists on the
planet \emph{Zarmina}.  The colonists unexpectedly find this planet to be home to life which rapidly becomes hostile.
Since it is a survival game, the story will not be expanded in detail.

\section{Gameplay}

The player will begin with 

Enemies will spawn offscreen and move towards the player's structures.  On reaching the structures, the enemies will damage them until they are eventually destroyed.

mineral deposits - must build towards and mine

\subsection{Structures}

There will be numerous types of structures available to the player, but not so many as to create confusion and a very steep learning curve.

\begin{tabular}{lp{9cm}}
Name & Description \\ \hline
Reactor & Provides the \emph{power} resource.  When all power cores are destroyed the game ends. \\
Foundation & Serves as a minor protective barrier and a platform on which to build other structures. \\
\end{tabular}

\subsection{Enemies}

There will be many varied types of enemies which will require different strategies for successful defense.

\subsection{Resources}

power, minerals

\subsection{Endgame}

There will be no win condition.  Instead, the objective will be to survive for as long as possible.  The game will end when all of the player's structures are destroyed.

\section{Technical features}

\subsection{Collision detection}

No sophisticated collision detection will be required.  Testing for when an enemy reaches a structure can be done adequately by thresholding Euclidean distance.  Enemies will elastically avoid each other, eliminating the need for precise collision detection between them.

\subsection{Artificial intelligence}

\end{document}
