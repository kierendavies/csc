\documentclass[a4paper,12pt]{article}

\begin{document}
\title{A Review of \emph{Dear Esther}}
\author{Kieren Davies}
\date{25 June 2012}
\maketitle

\emph{Dear Esther} is a single-player first-person adventure game developed by \emph{thechineseroom} with Robert Briscoe.  It was commercially released on 14 February 2012 on Steam for PC, although a previous incarnation has been available as a free mod for the \emph{Source} engine since 2008.

\emph{Dear Esther} offers around two hours of very unusual gameplay, notably lacking in interactivity.  The game begins by placing the player on the coast of an uninhabited island which the player is allowed to explore.  This is subtly guided by a narrative, presumably from the perspective of the player's character, although there are no clear goals.  Even as the game progresses, there is no interaction other than navigating the terrain of the island.

Freedom of choice is presented in a crude form: narrative is triggered by visiting certain areas, so it is up to the player to decide where to go to reveal parts of the story.  Ultimately this determines the order in which details are revealed.  Similarly, failing to explore everywhere results in some story elements remaining undiscovered.

One personal source of frustration was the slow movement

Despite (or perhaps because of) the above criticisms, \emph{Dear Esther} is a remarkable experience.

story - metaphors
scenery
  shiny
  budget - plants
soundtrack



\end{document}
