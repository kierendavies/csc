\documentclass[a4paper,12pt]{article}
%\usepackage{setspace}
%\onehalfspacing
\usepackage{aurical}
\usepackage[T1]{fontenc}

\begin{document}
\title{A Review of \emph{Dear Esther}}
\author{Kieren Davies}
\date{25 June 2012}
\maketitle

\begin{quote}
\centering \Fontlukas \large
The gulls do not land here any more; I've noticed \\
that this year, they seem to shun the place.
\end{quote}

\emph{Dear Esther} is a single-player first-person adventure game developed by \emph{thechineseroom} with Robert Briscoe.  It was commercially released on 14 February 2012 on Steam for PC, although a previous incarnation has been available as a free mod for the \emph{Source} engine since 2008.

\emph{Dear Esther} offers around two hours of very unusual gameplay, notably lacking in interactivity.  The game begins by placing the player on the coast of an uninhabited island which the player is allowed to explore.  This is subtly guided by a narrative, presumably from the perspective of the player's character, although there are no clear goals.  Even as the game progresses, there is no interaction other than navigating the terrain of the island.  Freedom of choice is presented in a crude form: narrative is triggered by visiting certain areas, and certain triggers prevent others, so it is up to the player to decide where to go to reveal parts of the story.  Ultimately this determines which details are revealed and in what order.  Considering all the above, some critics have claimed that \emph{Dear Esther} does not truly constitute a game.

Despite this, or perhaps because of it, \emph{Dear Esther} is a truly remarkable experience.  The story is incredibly poetic, and the narrative is delivered with passion.  What is particularly unusual is that plot and circumstances are only presented in the forms of various intertwined metaphors, integrated deeply with the visual environment.  It is up to the player to interpret these and draw his own conclusions---something which probably only appeals to more cultured gamers.  The story furthermore draws on ideas of loss, helplessness, and confusion, leaving the player unsettled, which can be somewhat thought-provoking when combined with the oblique nature of the whole game.

Since \emph{Dear Esther} is built on the \emph{Source} engine, it offers all the visual details and effects that PC gamers have come to know and love from games such as \emph{Half-Life 2}.  Nonetheless, there are some telltale signs of the small development budget; most noticeably, all the plants are 2D sprites which always face the player.  However, this is more than made up for by exceptional level design.  Every landscape in the game, be it a strip of beach or the inside of a shack or a cave scattered with glowing crystals, manages to be beautiful and still somehow very convincing.  The scenery changes gradually from purely natural to bizarrely fantastic, yet this is done so craftily that it is easy to remain totally immersed in it.

The atmosphere of the game is deepened by its incredible soundtrack, masterfully composed by Jessica Curry.  The music is somehow emotionally moving and yet not intrusive, each song is perfectly matched to the ambiance of some part of the story.

One minor aspect of gameplay which has a surprisingly significant impact is slow player movement.  At times it can be frustrating to have to spend so long travelling along a straight section of pathway, particularly for the more obsessive players who want to explore the island in its entirety.  On the other hand, this leaves ample opportunity to admire the scenery and reflect on the story, so it may not entirely be a bad thing.

To conclude briefly, \emph{Dear Esther} is one of those rare productions which redefines what a game can be which every discerning gamer ought to play.

\end{document}
